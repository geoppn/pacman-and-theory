\documentclass[12pt]{article}

% Set page size and margins
\usepackage[a4paper,top=2cm,bottom=2cm,left=2cm,right=2cm,marginparwidth=1.75cm]{geometry}

% Language setting
\usepackage[english,main=greek]{babel}
\newcommand{\gr}{\greektext}
\newcommand{\la}{\latintext}

% Page background & text colors
\usepackage[dvipsnames]{xcolor}
% define bg and txt colors
\definecolor{col_bg}{HTML}{FFFFFF} 
\definecolor{col_txt}{HTML}{000000}
% set the colors
\pagecolor{col_bg}
\color{col_txt}


% Useful packages
\usepackage{amsmath}
\usepackage{amsfonts}
\usepackage{amsmath,amssymb}
\usepackage{graphicx}
\usepackage{tabularray}
\usepackage{listings}
\usepackage{algorithm}
\usepackage{algpseudocode}
\usepackage{enumitem}
\usepackage[utf8]{inputenc}
\usepackage[T1]{fontenc}
\usepackage{csquotes}



\begin{document}
\begin{table}[ht]
    \begin{tblr}{
      @{}X[l,valign=b]X[c,valign=b]X[r,valign=b]@{}
    }

        % First line, course info
        \hline
        \SetCell[c=2]{l}{ΤΕΧΝΗΤΗ ΝΟΗΜΟΣΥΝΗ} & & {2024-25} \\ 
        \hline
        {} & {} & {} \\

        % Title
        \SetCell[c=3]{c}{ \Large \textbf{Εργασία 1} } \\
        {} & {} & {} \\

        % Info
        \SetCell[c=3]{c}{ \large \it Απαντήσεις  } \\
        {} & {} & {} \\
        
        % Name Surname, RN
        \hline
        \SetCell[c=3]{c}{ \textbf{Ονοματεπώνυμο: }  } Γεώργιος Παπαϊωάννου \\
        \SetCell[c=3]{c}{ \textbf{Α.Μ.:} 1115202100222 } \\
        \hline
        
    \end{tblr}
\end{table}
\section{} 
\subsection{\bold{\la DFS}}
Το {\la BFS} επεκτείνει όλους τους κόμβους μέχρι να φτάσει σε εναν στόχο (εδώ έχουμε βάθος = 4. Οι κόμβοι που επεκτείνονται σε κάθε επίπεδο είναι: \\
Επίπεδο 0: 1 κόμβος (ρίζα) \\
Επίπεδο 1: 3 κόμβοι (διακλάδωση ίση με 3) \\
Επίπεδο 2: $3^2$ = 9 κόμβοι, \\
Επίπεδο 3: $3^3$ = 27 κόμβοι, \\
Επίπεδο 4: $3^4$ = 81 κόμβοι (στόχος) \\
Συνολικά έχουμε 1+3+9+27+81=121.\\ Στό {\la DFS} έχουμε μικρότερο αριθμό κόμβων 41 (40 + 1 πρώτος κόμβος του βάθους 4) και μεγαλύτερο = 121.
\subsection{\bold{\la BFS}}
Το {\la DFS} επεκτείνει τους κόμβους κατά μήκος της αριστερότερης διαδρομής (εκφώνηση). Άρα έχουμε 2 περιπτώσεις \\
Αν ο στόχος βρίσκεται στον αριστερότερο κόμβο, επεκτείνονται μόνο 5 κόμβοι (ριξα - επ1 - επ2 - επ3 - επ4) αρα εινάι ο μικρότερος πιθανός αριθμός κόμβων. \\
Για τον μεγαλύτερο αριθμό, αν ο στόχος βρίσκεται στον δεξιότερο κόμβο στο βάθος 4, επεκτείνονται όλοι οι κόμβοι. Αρα 121. \\
\subsection{\bold{\la IDS}}
Στο {\la IDS}, επεκτείνονται οι κόμβοι για κάθε βάθος ξεχωριστά, αφού ο στόχος μας είναι στο βάθος 4, θα επεκειθούν όλοι οι κόμβοι στην χειρότερη περίπτωση αρα μεγαλύτερος αριθμός κόμβων = 121\\
Μικρότερος = 41 (πρώτος κόμβος που επισκέπτεται στο βάθος 4) .
\section{} 
\subsection{}
Συνολικό κόστος της βέλτιστης διαδρομής: 12.5
\subsection{}
45 κόμβοι.
\subsection{}
(0, 0), (0, 1), (1, 0), (0, 2), (2, 0), (0, 3), (3, 0), (1, 3), (1, 4), (1, 5), (3, 1), (4, 0), (1, 6), (2, 1), (2, 3), (2, 5), (3, 2), (4, 1), (1, 7), (2, 6), (3, 3), (3, 5), (2, 7), (4, 3), (4, 5), (4, 6), (4, 7), (0, 6), (0, 7), (2, 2), (4, 4), (5, 5), (5, 7), (3, 6), (6, 5), (6, 7), (7, 7), (0, 8), (3, 7), (4, 8), (5, 3), (6, 6), (7, 6), (8, 7), (9, 7)
\subsection{}
- Ευκλείδεια Απόσταση: Η ευκλείδεια απόσταση υπολογίζει την ευθεία γραμμή μεταξύ δύο σημείων, η οποία είναι η μικρότερη δυνατή απόσταση. Επομένως, δεν υπερεκτιμά ποτέ το πραγματικό κόστος της διαδρομής, ακόμα και όταν τα κόστη των κελιών είναι διαφορετικά. \\
- Απόσταση {\la Chebyshev}: Η απόσταση Chebyshev είναι η μέγιστη απόσταση μεταξύ των οριζόντιων και κάθετων αποστάσεων δύο σημείων. Σε ένα πλέγμα όπου επιτρέπονται διαγώνιες κινήσεις, αυτή η απόσταση αντιπροσωπεύει τον ελάχιστο αριθμό κινήσεων που απαιτούνται για να φτάσει κανείς από το ένα σημείο στο άλλο, ακόμα και όταν τα κόστη των κελιών είναι διαφορετικά.
\section{} 
\subsection{ Αναζήτηση πρώτα σε πλάτος και αναζήτηση περιορισμένου βάθους} \\
Η {\la BFS}: είναι πλήρης (εφόσον το {\la b} είναι πεπερασμένο). Η {\la DLS} δεν είναι πλήρης, επειδή περιορίζεται σε συγκεκριμένο βάθος και μπορεί να αποτύχει να βρει τον στόχο αν βρίσκεται σε μεγαλύτερο βάθος. Αρα το ζεύηος δεν είναι πλήρες λόγω της περιορισμένης αναζήτησης βάθους.  \\
η {\la BFS}: είναι βέλτιστη αν όλοι οι κόμβοι έχουν το ίδιο κόστος. η {\la DLS} δεν είναι βέλτιστη, καθώς μπορεί να διακόψει την αναζήτηση πριν βρει τη βέλτιστη λύση. ΤΟ ζεύηος δεν είναι βέλτιστο.

\subsection{ Αναζήτηση με Επαναληπτική Εκβάθυνση ({\la IDS}) και Αναζήτηση Περιορισμένου Βάθους (\la DLS)}
Η {\la IDS} είναι πλήρης για πεπερασμένο χώρο. Οπώς είπαμε, η {\la DLS} δεν είναι πλήρης. Άρα το ζεύηος δεν είναι πλήρης λόγω του περιορισμένου βάθους. \\
Η {\la IDS}: Είναι βέλτιστη αν όλοι οι κόμβοι έχουν το ίδιο κόστος. Η {\la DLS} δεν είναι βέλτιστη. Το ζεύγος δεν είναι βέλτιστο.

\subsection{ Α* και Αναζήτηση Περιορισμένου Βάθους (\la DLS)}
Η Α*: Είναι πλήρης αν η ευρετική συνάρτηση {\la h(n)} είναι επιτρεπτή. Η {\la DLS} δεν είναι πλήρης. Το ζεύγος δεν είναι πλήρης. \\ 
Η A*: Είναι βέλτιστη εάν η ευρετική συνάρτηση είναι επιτρεπτή και συνεπής. Η {\la DLS} δεν είναι βέλτιστη. Το ζεύγος δεν είναι βέλτιστο.
\subsection{ Α* και Α*} \\
Και οι δύο αναζητήσεις είναι πλήρεις, αρα το ζεύγος είναι πλήρες. \\
Και οι δύο αναζητήσεις είναι βέλτιστες για κάθε συνεπής συνάρτηση, αρα το ζεύηος είναι βέλτιστο. 
\end{document}
